\section{C-Einführung}
\subsection{Verständnis}
\begin{frame}
	\frametitle{\currentsection}
	\framesubtitle{\currentsubsection}
	\begin{block}<1->{Was ist ein Programm?}
		\begin{itemize}
			\item C Datei als Beschreibung des Programms
			\item Compilieren der C Datei mit Compiler zu Assembler
			\item Assembler stellen 1:1 die Befehle der CPU dar
			\item Übersetzen von Assembler nach Hex
			\item Aufspielen der Hex über Flasher
		\end{itemize}
	\end{block}
\end{frame}


\subsection{Funktionen}
\begin{frame}
	\frametitle{\currentsection}
	\framesubtitle{\currentsubsection}
	\begin{block}<1->{Was ist eine Funktion?}
		\begin{itemize}
			\item Strukturieren Programme
			\item Beschreiben den Einstiegspunkt (main Funktion)
			\item Sind ähnlich zu mathematischen Funktionen
			\item Können Werte zurückliefern aber auch Werte benötigen
		\end{itemize}
	\end{block}
\end{frame}
\begin{frame}
	\frametitle{\currentsection}
	\framesubtitle{\currentsubsection}
  \begin{block}<1->{Beispiele}
  	\lstinputlisting[language=C]{Code/Examples/function.c}
  \end{block}
\end{frame}


\subsection{Variablen}
\begin{frame}
	\frametitle{\currentsection}
	\framesubtitle{\currentsubsection}
	\begin{block}<1->{Wie merke ich mir Dinge?}
		\begin{itemize}
			\item Variablen sind dafür gedacht Werte zu speichern.
			\item Variablen haben IMMER einen Typ und einen Namen!
			\item Es gibt unter anderem folgende Typen:
			\begin{itemize}
				\item int, long
				\item float, double
				\item char
			\end{itemize}
		\end{itemize}
	\end{block}
\end{frame}
\begin{frame}
	\frametitle{\currentsection}
	\framesubtitle{\currentsubsection}
	\begin{block}<1->{Beispiele}
		\lstinputlisting[language=C]{Code/Examples/variable.c}
	\end{block}
\end{frame}

\subsection{Operationen}
\begin{frame}
	\frametitle{\currentsection}
	\framesubtitle{\currentsubsection}
	\begin{block}<1->{Beispiele}
		\lstinputlisting[language=C]{Code/Examples/operations.c}
	\end{block}
\end{frame}


\subsection{Verzweigungen}
\begin{frame}
	\frametitle{\currentsection}
	\framesubtitle{\currentsubsection}
	\begin{block}<1->{Wie entscheide Dinge?}
		\begin{itemize}
			\item Um Entscheidungen zu treffen gibt es
			\begin{itemize}
				\item if ([COND]) else
				\item switch
			\end{itemize}
		\end{itemize}
	\end{block}
\end{frame}
\begin{frame}
	\frametitle{\currentsection}
	\framesubtitle{\currentsubsection}
	\begin{block}<1->{Beispiele}
		\lstinputlisting[language=C]{Code/Examples/branch.c}
	\end{block}
\end{frame}


\subsection{Schleifen}
\begin{frame}
	\frametitle{\currentsection}
	\framesubtitle{\currentsubsection}
	\begin{block}<1->{Wie wiederhole ich Dinge?}
		\begin{itemize}
			\item Schleifen vermeiden redundanten Code
			\begin{itemize}
				\item z.B. Eine Liste durchgehen, ...
			\end{itemize}
			\item In C gibt es
			\begin{itemize}
				\item while ([COND])
				\item for (i=0; [COND]; i++)
			\end{itemize}
		\end{itemize}
	\end{block}
\end{frame}
\begin{frame}
	\frametitle{\currentsection}
	\framesubtitle{\currentsubsection}
	\begin{block}<1->{Beispiele}
		\lstinputlisting[language=C]{Code/Examples/loops.c}
	\end{block}
\end{frame}


\subsection{Compiler Stuff ... -.-}
\begin{frame}
	\frametitle{\currentsection}
	\framesubtitle{\currentsubsection}
	\begin{block}<1->{Includes}
		\begin{itemize}
			\item Funktionen/Dateien einbinden
			\begin{itemize}
				\item \#include ''somefoo.h''
				\item \#include <asuro.h>
				\item \#include <stdio.h>
				\item \#include <Math.h>
			\end{itemize}
		\end{itemize}
	\end{block}
	\begin{block}<2->{Defines}
		\begin{itemize}
			\item Variablen definieren
			\begin{itemize}
				\item \#define PI 3.14159265359
				\item \#define OUT "Some output"
				\item \#define SWITCH(X) 1<<X
			\end{itemize}
		\end{itemize}
	\end{block}
\end{frame}